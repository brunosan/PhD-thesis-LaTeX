\chapter*{Acknowledgements\markboth{Acknowledgements}{Acknowledgements}}
\addcontentsline{toc}{chapter}{Acknowledgements}
%\begin{comment}
%intro

I would like to acknowledge here the contributions from many people to the successful completion of this work. I want to express my gratitude to all those who helped me, from my supervisor this three last years to my teachers at my first university who encouraged me to start this adventure in astrophysics. During all this time I have been also surrounded and supported by many people to whom I want to thank within these lines. They all made this experience of doing a PhD, professionally and personally, one of the best times of my life.



%Franz
My supervisor of this research work was Prof. Dr. Franz Kneer, or better \emph{franz}. He was the first person I saw when I came to the train station back three years ago (along with Julian). He was the guide to all my work, the observations, the interpretation, \dots basically, my formation as solar physicist. The amount of knowledge I learnt from him cannot be measured by any means, and for that I am professionally extremely grateful. In a personal way he was also very kind and helped me always with good advices whenever I needed it. He understood me perfectly when I had problems and also encouraged me to travel as much as I wanted (which was not a little). Working with him this time made the whole experience of PhD leaps better. 
%aleman
Ich kann nicht diese Arbeit ohne Ihn zu vorstellen. Danke Franz.
%aleman

%family
Above all, I am and will be always grateful to my reference in life, my parents Conchita Nu\~no L\'opez and Julio S\'anchez-Andrade Fern\'andez, and my sister Deva S\'anchez Nu\~no. They always supported me, although  a bit far in physical distance during this last years. When I was 5 years old I was going home with my mother from the playground when I saw a poster announcing a conference about starts. I asked my mother to read it loud for me, and to her surprise I had to explained her: ``Don't you yet know that I want to be a \emph{researcher of stars}?''. The day after she gave me a children book about stars and explained me that if I wanted to be so I should know that they are called \emph{astronomers}. And that's how it all began.

%instituto
My work was supported by two institutions: the \emph{Max Planck Institute f\"ur Sonnensystemforschung} (MPS) granted me the fellowship and the \emph{Institut f\"ur Astrophysik G\"ottingen} (IAG) provided me the facilities to work with Franz at G\"ottingen and support for the observations at Tenerife. Also, being part of the \emph{International Max Planck Research School on Physical Processes in the Solar System and Beyond at the Universities of Braunschweig and G\"ottingen} (IMPRS) I could attend seminars, retreats and courses on various astrophysical topics. I am very grateful to these institutions for providing such a broad curricula. I would like also to thank the coordinator of the MPS and IMPRS, Dr. Dieter Schmitt. 

%other institutions
The Vacuum Tower Telescope used for the observations is operated by the Kiepenheuer-Institut f\"ur Sonnenphysik, Freiburg, at the Spanish Observatorio del Teide of the Instituto de Astrof\'isica de Canarias.  

%proffesional help (IAG + otros)
Here at the solar group of IAG we had a really stimulating environment with long discussions: Franz, Klaus, Markus, Nazaret, Julian and the various Erasmus people that went by (like Luisa, Manu, \dots ). We worked in the beginning at the beautiful historical \emph{Sternwarte} and then at the Faculty of Physics. I would like to thank specially Klaus for all the help and scientific discussions during all the time he was here.

Many other professional colleagues contributed to this work. Their input was extremely helpful. A short list with few names would include: (from the MPS) Andreas Lagg, Vasili Zakarov, (from the IAG) the solar group, Axel Wittman, Volker Bothmer, (from the IAC) Basilio Ruiz Cobos, Manolo Collados, Javier Trujillo, Rebecca Centeno. Thanks to Nurol Al and H. Schleicher for their computing codes. For the Blind Deconvolution section I got much help from Jaime and Mats L\"ofdahl at the Institute for Solar Physics, Stockholm.

%otros profesionales
From Asturias to here there is long way, specially passing through Canary Island. Without the advices and guidance from many people I would have been lost. In these sense I specially thank Cristina, Juanjo, Basilio and Fernando. 

%amigos
Throughout all these years I had the luck to be surrounded by the best companion of friends. I would like to mention lastly some few friends that helped me to keep my feet on the ground. Here in G\"ottingen Thorsten, Gonzaga, Klaus, Miguel, Cristiano, Vladi, Cathi, Niko, Lorne, Diego, Iria, Teresa, Nora, Carlos, Olga, Julian, Nazaret, Benoit, Markus, Manu and Luisa. Thorsten in particular was not only a good friend but also my flat-mate, climbing-mate, snowboard-mate and of course party-mate. My new flat-mate, Richard, had to stand me this last months with my thesis mood, though. Thanks a lot!.  Mark, Lucas, Clementina, Pedro, Emre or Martin, all the guys from Lindau, always warmly welcomed his lost member of the family emigrated to the civilization.

During my short intense period in Tenerife I made good friends: Onti, Borja, Bendinat, Adriana, Manu, Miguel. Some time later I met Rebecca, with whom I also worked these years.

Whenever a crazy trip came up I had AEGEE people coming along. They would join me to any place in Europe. With them I had some of my best times: Luis, David, Neila, Saioa, Javiero, Marti, Marta and recently of course Carol. All those friends with crazy names: Annia, Annamary, Bir, Konstantina, Marios or Lutzn. I am sure I'll meet you again, somewhere.

Since I left Asturias, I have always missed the green and astounding landscapes, but also my old friends: Roberto, Raul, Estela, Flaci, Nacho, Cova, Lisa, David and of course my big and warm family.

%final
Por ultimo me gustar\'ia terminar esta secci\'on, y con ello el fin de este trabajo, con la memoria de tres personas a quienes siempre echar\'e de menos. En mi primer a\~no en Alemania tuve que despedir a  mi abuela materna y el segundo a mi abuela paterna. Aunque sea ley de vida no deja de ser doloroso. Este \'ultimo a\~no, a mi primo Abraham, a quien tuve la suerte de ver una \'ultima vez. Con ellos he aprendido la lecci\'on m\'as importante:

\vspace{1cm}
{\emph{ \centering Disfruten de la vida.}}
%\vspace{3cm}

%\end{comment}

\flushright{
G\"ottingen, January 2008} 
