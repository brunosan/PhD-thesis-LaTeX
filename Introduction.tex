\chapter{Introduction}

This thesis deals with the chromosphere of the Sun. To give some insight to the readers which are not familiar with the topics of this work we introduce in Section \ref{intro:sun} the main characteristics of the Sun with a short general description. This will elucidate the position of the chromosphere in the solar structure and its role for the outer solar atmosphere. In the subsequent Section \ref{intro:chromo}, those aspects of the chromosphere which are treated in the present work are specified. Finally Section \ref{ref:out} indicates the structure of this thesis work.
 

\section{The Sun\label{intro:sun}}
\begin{flushright}
\emph{It is just a ball of burning gas\\ \dots right?\\
\vspace{1cm}}
\end{flushright}

The Sun is the central object of the Solar System, which also contains planets and many other bodies such as planetoids (small planets), comets, meteoroids and dust particles. However, the Sun on its own harbors 99.8\% of the total mass of the system, so all other objects orbit around it. 

The Sun itself orbits the center of our Galaxy, the \emph{Milky Way}, with a speed of $217$ km/s. The period of revolution is $\sim230$ million years (the last time the Sun was on this part of the Galaxy was the time the Dinosaurs appeared). Compared to the population of stars in our galaxy, the Sun is a middle-aged, middle-sized, common type star. In astrophysicist's language it is of spectral type \emph{G2} and of luminosity class \emph{V}, located on the main sequence of stars in the Hertzsprung-Russell diagram. According to our understandings derived from models, it has been on the main sequence for $5\,000$ million years and it will remain there for another $5\,000$ million years before starting the giant phase.

The Sun is the closest star to us, the next one being $250\, 000$ times further away, but still light from the Sun's surface takes around 8 minutes to reach the Earth. It is the only star from where we get enough energy to study its spectrum in great detail and with short temporal cadence. With indirect methods, we can produce images of the surface structuring on other nearby starts. But on the Sun, with current telescopes and techniques, we resolve structures down to 100 km size on its surface, which represents approximately the resolution limit in this thesis work. We can also investigate the structure of its atmosphere and the effects of its magnetism. Actually, we are embedded in the solar wind that has its origin in the outer solar atmosphere, the corona of the Sun. Thus, we can make \emph{in-situ} measurements.  With special techniques and models, we can reconstruct the properties of its interior.
% franz I refer to helioseismology, for example.
\pagebreak
\begin{wrapfigure}[14]{r}{0.5\textwidth}
\vspace{0.5cm}
\begin{center}
\includegraphics[width=0.5\textwidth]{../figures/sol.jpg}
\caption{The apparent size of the Sun on the sky is $\sim32 ' $, a little bit larger than one half degree. }
\label{fig:foto:sol}
\end{center}
\end{wrapfigure}

The Sun is the most brilliant object in the sky, 12 orders of magnitude brighter than the second brightest object, the full Moon, which actually only reflects the sunlight.   Its light warms the surface of the Earth and is used by plants to grow. Its radiation is the input for the climate. The solar wind separates us from the interstellar medium. The magnetism of the Sun protects us from cosmic high-energy radiation and it influences the climate on Earth. Violent events in the solar ultraviolet radiation and the solar wind can also disrupt radio communications.

The  Sun possesses a complex structure. Essentially, it can be described as a giant conglomerate of Hydrogen and Helium ($\sim74$\% and $\sim24$\% of the mass, respectively) and traces of many other chemical elements. Due to its big mass the self-gravitation keeps the structure as a sphere. From the weight of the outer spherical gas shells the pressure increases towards the center of the sphere. During the gravitational contraction of the pre-solar nebula towards its center, i.e. when forming a protostar, the gas has heated up by converting potential energy into thermal (kinetic) energy. This produces, together with a high gas density, a high pressure, which prevents the sphere from collapsing further inwards. Eventually, near the center, temperature and pressure are high enough to ignite nuclear reactions.
\subsubsection*{Structure}
At the core of the Sun the density and temperature (of the order of 13 million Kelvin, or $13 \cdot 10^{6}$ K) are high enough to fuse hydrogen and burn it into helium. This process also produces energy in the form of high-energy photons.  This continuous, long-lasting energy output from the nuclear reactions keeps the core of the Sun at high temperature to sustain the gravitational load from the outer gas shells. Due to the high density, the photons are continuously absorbed and re-emitted by nearby ions, and in this way the big energy output is slowly \emph{radiated} outwards,  while, towards the surface of the Sun, the density decreases exponentially, along with the temperature. Photons reaching today the Earth's surface were typically generated on the early times of \emph{Homo Sapiens}, as the typical travel time is $\sim170\,000$ years \citep{1992ApJ...401..759M}. 

At a distance from the center of approximately 70\% of the solar radius, the radiation process is not efficient enough to transport the huge amount of energy produced in the core. There the gas is heated up, and expands, it becomes buoyant and rises. This creates \emph{convection} cells in which hot material is driven up by buoyancy while cool gas sinks to the bottom of the cells, where it is heated again. These gas flows transport the energy to the outer part of the Sun, where the temperature is measured to be $\sim 5\,700\,$K and the density is low enough that the photons can escape without much further absorption. The outer region from where we receive most of the optical photons can be called the surface of the Sun, although it is not a layer in the solid state. It is called the \emph{photosphere} (sphere of light). Most of the photons we receive come from this layer are in the \emph{visible} part of the spectrum: light. This is why Nature favored in the late evolution process the development of vision instruments that are more sensitive %
in the spectral region in which most emission from the Sun occurs.
\begin{figure}[t]
\begin{center}
\includegraphics[width=\textwidth]{../figures/mosaicb.png}
\caption{High resolution image of the ``surface'' (photosphere) of the Sun with a resolution of $\sim140$ km. Granules are seen all around the photosphere outside the dark areas. They form the uppermost layers of the convection zone, in which the energy is transported from deep down outwards via gas motions. At the top, the gas cools down by radiating photons into space. Localized strong magnetic fields can also emerge and are seen as dark areas, the sunspots, which are a consequence of the less efficient energy transport.}
\label{fig:photosphere}
\end{center}
\end{figure}

Further out of this layer the atmosphere of the Sun extends radially, with decreasing density. In this outer part, with its low density, magnetic fields rooted inside the Sun cease to be pushed around by gas flows. This transition occurs together with a still not completely understood increase of temperature up to several million degrees. Therefore,  there must be a layer with a minimum temperature. Standard average models place it at a height of about 500 km with a temperature of about 4000 K, which is low enough to allow the formation of molecules like CO or water vapor. Beyond this layer the temperature rises outward. Again in standard models, the layer following the temperature minimum has an extent of about 1\,500 km and its temperature rises to 8\,000 -- 10\,000 K. This layer is called the \emph{chromosphere}. The present work deals with some of its properties. Outside the chromosphere, the temperature rises abruptly within the \emph{transition region}. The outermost part of the atmosphere, called \emph{corona}, drives a permanent outwards flow of particles moving along the magnetic field lines. This \emph{solar wind} extends to $100\,000$ times the solar radius, far beyond Pluto's orbit, to the outer border of our Solar System, the \emph{heliopause}. There the interaction with the interstellar medium creates a shock front, which is being measured these years by the Voyager 1 and Voyager 2 probes.

Beyond this layered structure, the Sun is far more complex. Some other properties, which we describe shortly, are: 

- The Sun vibrates. As a self gravitating compressible sphere, it vibrates. Pressure and density fluctuations mainly generated by the turbulent convection, are propagated through the Sun. Waves with frequencies and wavelengths close to those of the many normal modes of vibration of the Sun add up to a characteristic pattern of constructive 
%?doppler shifts of p modes are about this order, rigth?
interference. This vibration, although of low amplitude with few 100 m/s in the photosphere, can be measured and decomposed into eigenmodes by means of Doppler shifts and observations of long duration. The propagation of the waves depends on the properties of the medium. It is possible then to infer these properties from the measured vibration patterns. Some waves propagate only  close to  the surface, but others can propagate through the entire Sun. These latter waves provide means to infer some structural properties, such as temperature, of the solar interior and test models of the Sun.  
%franz i remove it : Nonetheless the velocitiy of this waves depends on the density, meaning that they carry less information of the inner parts. 
\emph{Global Helioseismology} provides means to infer the global properties of the interior of the Sun studying the vibration pattern, while \emph{local helioseismology} can depict the surroundings of the local perturbations.

- The Sun rotates. The conservation of angular momentum of a slowly rotating cloud that will form a star result, upon contraction, a rapid rotation. It is commonly accepted that most of the Sun's angular momentum was removed during the first phases of the life of  the Sun by braking via magnetic fields anchored in the surrounding interstellar medium and by a strong wind. The remaining angular momentum leads to today's solar rotation period. But being the Sun not a rigid body this rotation varies  from layer to layer and with latitude. Gas at the equator rotates at the surface with a period of 27 days, faster than at the poles where the rotation period is approximately 32 days. Using helioseismology observations we know that this differential rotation continues inside the Sun, until a certain depth, from which on the inner part rotates like a rigid sphere with a period of that at middle latitudes on the surface. This region corresponds to the layer where the convection starts, at around $0.7$ solar radii, and is called the \emph{tachocline}. The differential rotation creates meridional flows of gas directed towards the poles near the surface and towards the equator near the bottom of the convection zone.

- The Sun shows (complex) magnetic activity. The Sun possesses a very weak overall magnetic dipole field. However, the solar surface can host very strong and tremendously complicated magnetic structures, which can be seen through their effects on the solar plasma, e.g. less efficient energy transport (that leads to dark sunspots). All matter in the Sun is in the form of plasma, due to the high temperature.  The high mobility of charges that characterizes the plasma state, makes it highly conductive, causing magnetic field lines to be "frozen" into it. Provided that the gas pressure  is much higher than the magnetic pressure, the magnetic field lines follow generally the dynamics of the plasma. The source of these localized strong magnetic fields is still to be understood. The dynamo theory addresses this problem suggesting that the weak dipolar magnetic field is amplified at the bottom of the convection zone by the stochastic mass motion and shear produced by the convection and the differential rotation.

- The Sun has cycles. The Sun suffers fluctuations in time. Changes occur in the total irradiance, in solar wind and in magnetic fields. They happen in approximately regular cycles, like the 11 years sunspot cycle, and aperiodically over extended times, like the Maunder Minimum (a period of 75 years in the XVII century when sunspots were rare, and which coincided with the coldest part of the \emph{Little Ice Age}). These fluctuations modulate the structure of the Sun's atmosphere, corona and solar wind, the total irradiance, occurrence of flares and coronal mass ejections and also indirectly the flux of incoming high-energy cosmic rays. None of these variations are fully understood and their effect on the Sun itself or Earth is still under debate.  The generally accepted idea about the cyclic and more aperiodic fluctuations is that they are caused by variable magnetic fields. These are generated by dynamo mechanisms.

- The Sun evolves. The Sun is now in its main-sequence phase, where the main source of energy is the nuclear fusion of hydrogen to helium. After the initial phase of accretion of mass, a self gravitating star enters this phase, which lasts for most of its life. In the case of the Sun this phase will continue for approximately another five million years, after which the later evolution stages include a complex variation of the radius, with burning of helium as the source of energy in a later red giant phase. After this stage, the mass of the Sun is believed to be not large enough to undergo further fusion stages, and the Sun will slowly faint as a white dwarf star. 


Readers can find further general information about the Sun in e.g. \cite{wikisun,Stix:2002lr} and many references therein. 

\section{The chromosphere\label{intro:chromo}}

In our short description of the Sun's structure we stated that the atmosphere of the Sun comprises a layer above the photosphere in which the temperature begins to rise again until the transition region where an abrupt increase of temperature, from approximately 10\,000~K to 1 million K, occurs. This first layer above the photosphere is called \emph{chromosphere}. The name comes from the greek of ``color sphere'', as it can be seen as a ring of vivid red color around the Sun during total solar eclipses\footnote{The apparent size of the Sun on the sky happens to be very similar to the apparent size of the Moon, leading to annular or total solar eclipses, during which the red ring can be seen.}.

The boundaries of the chromospheric layer are very rugged, resembling more cloud structures than a spheric surface. Above quiet Sun regions the chromosphere can be about 2\,000 km thick, but some structures seen in typical chromospheric lines can reach to much higher altitudes, like filaments (that can reach heights of $350\,000$ km). 

The solar chromosphere is a highly dynamic atmospheric layer. At most wavelengths in the optical range, it is transparent due to the fact that its density is low, much lower than in the photosphere below it.  Nevertheless, in strong lines like H$\alpha$ (at 6563 \AA) or \ion{Ca}{II} K and H (at 3934 \AA\, and 3969 \AA, respectively) we have strong absorption (and re-emission) which allows direct studies about its peculiar characteristic, like bright plages around sunspots, dark filaments across the disk, as well as spicules and prominences above the limb. Indeed, recent works, e.g. \cite{2003A&A...402..361T}, suggest that many of these chromospheric features could all have the same physical properties but within different scenarios. 
\begin{figure}[t]
\begin{center}
\includegraphics[width=\textwidth]{../figures/mosaicn.png}
\caption{High resolution filtergram taken in the center of the H$\alpha$ spectral line, showing the chromosphere of the Sun with an image resolution of $\sim150$ km. The same field of view as image \ref{fig:photosphere}. The localized strong magnetic fields causing sunspots in the photosphere are seen now as fibrils around the sunspots. Given the low $\beta$ parameter, the plasma is forced to follow the magnetic lines, providing visible tracers and the variety of structures seen in the chromosphere. In the image we can see a  carpet of spicules, plage region and a top view of a rising twisted magnetic flux tube above the active region. This image corresponds to the dataset ``sigmoid'' studied in Chapter \ref{chapter:hr}.}
\label{fig:chromosphere}
\end{center}
\end{figure}

The temporal evolution of the chromospheric structures is complex. The dynamics of a magnetised gas depends on  the ratio of the gas pressure  $P_\mathrm{gas}$ to the  magnetic pressure $P_\mathrm{mag}$, i.e. the plasma $\beta$ parameter, $\beta$\,=\,$P_\mathrm{gas}/P_\mathrm{mag}$, with
$P_\mathrm{mag}$\,=\,$B^2/(8\pi)$ and $B$ the magnetic field strength \footnote{It is very common in astrophysics, specially in solar physics, to use magnetic field strength synonymously with magnetic flux density. The reason is that in most astrophysical plasmas B=H in Gaussian units. We follow this use in this thesis.}. From the low chromosphere into the extended corona, this plasma parameter decreases from values $\beta>1$, where the magnetic lines follow the motion of the plasma (as in the photosphere and solar interior) to a
low-beta regime, $\beta\ll1$, where the plasma motions are magnetically driven, and the plasma follows the magnetic field lines, creating visible tracers of the magnetism. These effects give rise to a new variety of energy transport and phenomena, like magnetic reconnection, filaments standing high above the chromosphere or erupting prominences.

%\begin{comment}
%\clearpage
\section{Aim and outline of this work\label{ref:out}}
Since the discovery of the chromosphere and since the hand-drawings of \citet{secchi1877} we have been able to observe this solar atmospheric layer in much detail. Many theoretical models have been proposed to understand its peculiar characteristics. But, only in the last recent years we have been able to address the problem with fine spectropolarimetry and high spatial resolution. We can study the fine details and resolve small structures, following their dynamics in time. Within these recent advances it has been possible  both to test current theories and to observe new unexpected phenomena. This work thus aims at contributing to the understanding of the solar chromosphere.

This first Chapter provided a broad introduction to the context of this work. We have briefly presented some general properties of the Sun and the chromosphere. In the following pages, throughout Chapter 2, we summarize some theoretical concepts of radiative transfer and spectral line formation needed for this work. We also present general characteristics of the two spectral lines studied: H$\alpha$ and \ion{He}{i} 10830\, \AA. Chapter 3 presents in detail the observations. There we also summarize the characteristics of the used telescope and optical instruments, as well as the data reduction and post-processing methods applied to achieve spatial resolutions better than $0\farcs5$.
Next, in Chapter 4, we discuss results from data on the solar disc, dealing with the chromospheric dynamics and fast events observed in our data. We present the observations of magnetoacustic waves as well as other fast events.  Chapter 5 is devoted to the spicules above the solar limb. The analysis of the spectroscopic intensity profiles from spicules in the infrared spectral range can be used to compare current theoretical models with observations. Further, we present  high resolution images in H$\alpha$ of spicules. Finally, the concluding Chapter 6 of this thesis summarizes the main conclusions and gives an outlook for future work.

%\end{comment}

%%

